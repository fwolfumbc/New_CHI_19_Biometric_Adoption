% \section{Limitations}
% 86ing this section and moving key point into Methods
% While the semi-structured interview-based methodology chosen for this this study is appropriate for an exploratory qualitative inquiry, its inherent issues merit review. 

% \subsection{Potential Demographic Effects}
% % 86ing this section and moving expert gender comparison survey to Participant Demographics and Sampling section in Methods 
% Firstly, we ask if the characteristics of biometric authentication are distinct between everyday users and those with a more security informed background. We have tried to be diligent in stating that the important distinctions between the two user groups may be due to that security awareness (the users themselves may often directly attribute their behavior to it), but other possible effects should be acknowledged. We tried to demographically balance the two samples for age and gender, but the expert cohort is somewhat older and more male, which may influence responses on some subjects. A cybersecurity industry survey from 2016 found women employed in the field at a rate 11\%, which roughly aligns with the 21.1\% female gender balance of our expert cohort (although employment alone in that specific field was not a required attribute) \cite{cyber2016}, such as information sharing \cite{das2018breaking}. It is possible that is happening here as well. Further research utilizing additional methodological approaches will be necessary to establish this.  

% \subsection{Potential Hawthorne and Recall Effects}
% %To save space, 86ing this whole section and instead adding mention of countering Hawthorne effects to the Methods section as part of justifying the question design (overlapping, open-ended, etc.)
% Secondly, as in most interview or survey-based approaches, we see the possible impact of Hawthorne effects. No expert or non-expert wants to look foolish, and asking about security habits is a subject that may cause participants to hide or embellish their answers to avoid looking uninformed. We have tried to counter this with open-ended and overlapping question design and ad hoc follow-up questioning that required participants to thoroughly explain their responses. Additionally, we acknowledge that a number of the interview questions and our derived observations are based upon recall-based responses that involve participants estimating when they took actions and how long they carried out behaviors in the past. We have treated these estimates as approximate, given that such recall is unlikely to be consistently accurate. 
