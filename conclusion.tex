\section{Conclusion}
% Existing research on expert mental models of mobile security has generally found
% the transferability of behavioral observations to everyday users to be fairly be
% limited. Experts' expectations of security features either differed due to their
% more complex picture of underlying technology dependencies and risk, or they
% simply chaffed against security best practices in a similar way to everyday
% users.

We have offered here a comparative picture of expert and non-expert adoption of
biometric authentication methods (primarily fingerprint recognition), with a
detailed explanation of what motivates differences between the two user
groups. For most of these users, this is a relatively recent adoption process,
and the experts involved have offered a picture of the bargain they have struck
between their long-standing awareness of network computing risk and their desire
for better mobile computing usability. At the same time, we were able to gather
and compare initial perspectives of these users, already accommodated to
fingerprint registration, towards facial recognition as it is being made more
broadly available on consumer devices. Based upon these findings (summarized in
Table \ref{tab:overview}), we have also presented the implications for effective
biometric authentication posed by the issues raised by the participants. Several
distinct points of misunderstand and mistrust regarding biometric authentication
were made clear, and we offer insight on how these can be addressed. Improvement
along these lines can be expected to also ease users' acceptance of
biometric-controlled application use.

%%% Local Variables:
%%% mode: latex
%%% TeX-master: "main"
%%% End:
