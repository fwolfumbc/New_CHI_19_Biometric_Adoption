\section{Related Work}

\paragraph{Rise of Biometrics}
Securing access to mobile devices is an increasingly urgent priority. %given the amount of personal and sensitive information stored or access through these technologies. 
Initiatives such as bring your own device (BYOD) or bring your own technology (BYOT), as well as the need to authorize mobile apps and purchases, have compounded the need for greater levels of security in both and home and work. Biometric authentication, including recognition of a user's fingerprint, face, or voice, addresses this need, with estimates suggesting all mobile devices will be mobile-enabled by 2020 \cite{biometricsupdate}. %Examples of user authentication techniques include recognition through the user's fingerprint, face, and voice (e.g. VoicePIN). 
Other biometric methods include continuous authentication (e.g. keystroke dynamics over a period of time) \cite{meng2015surveying}. %These offer fast methods of unlocking technologies, but can also be used to access apps, make payments etc.  
Research indicates user acceptance, including security and usability, is a major concern in biometric adoption \cite{rashed2015towards}. Recognizing this, manufacturers have published descriptions of underlying technical protections such as the Apple Secure Enclave to assure users \cite{iOS_security}. %These descriptions address underlying technical protections such as the Apple Secure Enclave, which may serve broadly to reassure skeptical users of the integrity of biometric implementation.   

\paragraph{Attitudes Towards and Adoption of Biometrics}
Biometric adoption has been studied from a number of perspectives. Lab-based studies have compared users' task performance and perception of different unlocking methods between operating systems, finding complex interaction between biometric method and users' task accuracy and acceptance \cite{bhagavatula2015biometric,trewin2012biometric}. 
%Trewin et al. also compared unlocking speed, error rate, recall, and acceptance of of voice, face, and gesture recognition. Speaking a PIN number with voice recognition was found to be the fastest method, with both face and voice recognition being faster than regular passcode entry. Users, however, found the biometric methods to have low usability and combinations of biometric methods were also found to be errorful and unpopular \cite{trewin2012biometric}.

Lab and focus group studies have also surveyed user opinion directly on biometric issues such as acceptance of continuous authentication %Crawford and Renaud asked lab participants to carry out several tasks on a mobile phone at varying levels of data sensitivity. Users were then surveyed on their willingness to accept transparent authentication methods that would replace conventional unlocking with continuous behavioral measures. 90\% of the 30 users were favorable towards accepting the transparent unlocking methods but their views varied widely regarding its suitability to control access to more sensitive tasks. These results suggested that while removing the usability headaches of unlocking with biometrics was desirable, more granular controls over access to specific functions would be advantageous and could better inform users' mental model of their phone's operation 
\cite{crawford2014understanding} and securing sensitive transactions %Karatzouni et al. found a positive outlook in focus groups towards biometrics as a means to secure increasingly sensitive mobile transactions 
\cite{karatzouni2007perceptions}. Additional facets of biometric adoption examined by researchers include capturing user motivation in specific geographic regions \cite{rashed2015towards}, and the impact of romantic relationship status on password and account sharing \cite{park2018share}.
%regional motivations comparative have also examined Rashed and Alajarmeh looked more specifically at the motivation towards biometric adoption of Arab mobile phone users, finding perceived usefulness to be of primary concern, superseding ease of use and security. Interestingly, age was also found to correlate with willingness to try new authentication methods \cite{rashed2015towards}. Park et al. used an online survey of romantic relationship status and password and account sharing to generate a taxonomy of sharing motivations and behaviors. 

The underlying technical integration of biometric unlocking has also been studied, in terms of potential attack points and countermeasures \cite{meng2015surveying}, and the security of API implementation \cite{bianchi2018broken}.
%technology has also been studied closely. Meng et al. looked at eleven types of physiological and behavioral biometric recognition types, in terms of their potential attack points and countermeasures, and provided a generic framework for evaluation \cite{meng2015surveying}. Similarly, Bianchi et al. examined the Android fingerprint recognition API, its use in a wide selection of applications, its resiliency against relevant threat models, and its documentation. ''Worrisome'' issues were found; 53.7\% of the apps reviewed did not include full checks as to whether a user actually touched the device's reader, only 1.8\% fully utilized the most secure implementation, and best practice examples in the documentation were incomplete \cite{bianchi2018broken}.

\paragraph{Perceptions of Individuals with Security Expertise}
Most directly related to this study, research has also examined the prevalence of mobile security awareness and its influence on behaviors such as authentication adoption. These studies indicate that security-informed users are a significant population with needs worthy of investigation.%, and a window into concerns that may be transferring to the broader mobile user base as security knowledge becomes more commonplace. 

%>>> HEY IF WE MOVE SURVEY RESULTS INTO INTRO THEN RE-WORD THIS PART ABOVE ABOUT TRANSFERABILITY (support for assertion isn't in this section anymore) <<<
%
% *** This section talk about why expert behavior might be good to examine so leaving the lengthier description, but focusing on the expert effects.

Researchers have examined differences and similarities in security expert and non-expert mental models of the information technology, and how these views may influence behavior. Kang et al. %looked at motivation for security conscious behavior in IT-expert and non-expert users. 
found advanced mental models of Internet processes in IT experts imparted more awareness of privacy risks, but did not translate into more secure behavior versus non-experts \cite{kang2015my}. Similarly, Friedman et al. surveyed Internet users from rural, suburban, and high-tech sectors about web security, and found users from all three groups poor comprehension of security features \cite{friedman2002users}. Stobert and Biddle used thematic analysis of semi-structured interviews to describe academic and industrial security experts' (n=15) password management. These users were found to vary between laxness and caution based on perceived risk towards their sensitive accounts \cite{stobert2015expert}.

Research has described several effects that security knowledge may impart. Das et al. %used just-in-time surveys following security and privacy-related news events (such as data breaches) to 
surveyed the influence of demographics and security behavioral intention (SBI) on news sharing after major security-related events. %SBI was among several factors affecting sharing behavior outcomes.
High SBI correlated with consuming security news online and willingness to share news generally, and specifically with colleagues and significant others. The authors were surprised to report correlation between low SBI and willingness to share security and privacy news because the participant noticed others behaving insecurely \cite{das2018breaking}.

%The type of news event affected the rate of sharing (financial breaches were most often shared), along with several demographic facets that impacted sharing and how participants learned about security and privacy news. Older participants were less likely to learn of news events through conversation, and men reported a greater sense of obligation to share. 

Huh et al. also found that prior security awareness played a role in adoption of mobile features. An online survey of 349 mobile tap-and-pay users and deliberate non-users found usability to be the paramount consideration. However, deliberate non-users of the feature cited security as their main deterrent. In that group, there was often misunderstanding of how credit card information was stored on the mobile phone and shared during transactions. A correlation was also found between security knowledge and adoption of the tap-to-pay feature \cite{huh2017don}.

Biometric adoption has been studied, but not with a comparative approach to those with security training. These expert related studies indicate that security-informed mental models may be expected to differentiate some mobile user choices, but experts and everyday users may both be subject to misunderstanding and not adhere to rigorous security practices. We have examined this cohort alongside everyday users and describe a number of contrasting themes in their adoption practices.  
% Researchers have examined how security expert and non-expert mental models of the information technology and the Internet differ. Kang et al., looked at motivation for security conscious behavior in IT-expert and non-expert users. Advanced mental models of Internet processes imparted more awareness of privacy risks but did not translate into more secure behavior \cite{kang2015my}. Similarly, Friedman et al. surveyed Internet users from rural, suburban, and high-tech sectors about web security features. Users from all three groups poorly understood and explained the security features and their own models of them \cite{friedman2002users}.
% Stobert and Biddle used thematic analysis of semi-structured interviews to describe academic and industrial security experts' (n=15) password management. These users were found to split their approach between laxness and caution in password use, based on awareness of risk towards their more sensitive accounts \cite{stobert2015expert}.

% Das et al. used just-in-time surveys of 1999 participants following twenty security and privacy-related news events (such as data breaches) to examine the influence of demographic factors and security behavioral intention (SBI) with news sharing. Several factors affected sharing outcomes. The type of news event affected the rate of sharing (financial breaches were most often shared), along with several demographic facets that impacted sharing and how participants learned about security and privacy news. Older participants were less likely to learn of news events through conversation, and men reported a greater sense of obligation to share. High SBI was found to correlate with hearing security and privacy news through online sources (along with male and younger participants), and willingness to share news generally, and specifically with colleagues and significant others. The authors were surprised to report correlation between low SBI and willingness to share security and privacy news because the participant noticed others behaving insecurely \cite{das2018breaking}. Huh et al. also found that prior security awareness played a role in adoption of mobile features. An online survey of 349 mobile tap-and-pay users and deliberate non-users found usability to be the paramount consideration. However, deliberate non-users of the feature cited security as their main deterrent. In that group, there was often misunderstanding of how credit card information was stored on the mobile phone and shared during transactions. A correlation was also found between security knowledge and adoption of the tap-to-pay feature \cite{huh2017don}.


% \paragraph{Potential transferability of security-informed user behavior to everyday users}
% --- MOVING TO INTRO
% Conclusions about the usability of mobile authentication for security trained users may not transfer easily to untrained users. A 2013 industry survey found that as much as 64\% of everyday users do not use any screen locking method at all, which would be difficult to reconcile with users steeped in threat awareness \cite{Consumer_survey2016}. Further, usability needs related to authentication and security between these two groups have been found to diverge \cite{asgharpour2007mental,bravo2011bridging,schaub2012password,von2013patterns,whitten1999johnny}, or to be largely unsatisfied for both groups by traditional unlocking methods \cite{friedman2002users,ion2015no,kang2015my}. However, there is cause to investigate changes in this relationship, as broader media exposure of data risk and security may stoke more concern, even among everyday users. 
% For example, a 2016 survey found 64\% of 18 to 26-year-old Americans recalled seeing news of a cyberattack in that year, almost doubling the rate of the previous year. 53\% of respondents in the United States also reported that cybersecurity policy influenced their choice of political candidates. Worldwide over the same period, 59\% of male young adults and 51\% of female young adults reported receiving formal cyber security training \cite{Raytheon_NCSA_survey2016}. In the United States, awareness of cybersecurity issues is also fueled by job opportunities in that field. The United States Bureau of Labor Statistics projected 18\% growth in information security jobs from 2014 to 2024, compared to 12\% for IT generally and 7\% for all types of employment \cite{BLS_jobStats2015}.  


%\paragraph{Observing recent and ongoing biometric adoption trends}
% ----- MOVING THIS TO INTRO
%Given the research on these subjects, we can expect to gain insight into mobile authentication adoption under differing levels of concern and understanding between everyday and security-conscious users, and differences and similarities in their behavior. However, we are further motivated to specifically examine adoption of biometric unlocking. This is because fingerprint recognition, while a relatively new offering by major technology providers (the most common fingerprint unlocking method in our cohort, iOS TouchID, is 5 years old at the time of this study), at the time of this writing has gained user acceptance. As a result, many security conscious users will have recent memory of how they struck their own bargain with any doubts about its use. Concurrently, at the time this study was conducted, facial recognition is also becoming a common feature in newer mobile phones. Like fingerprint unlocking, this feature offers convenience and usability, while potentially re-triggering anxiety over security data compromise. Essentially, the timing is good to use biometrics as a subject for observing technology adoption behavior under differing levels of concern and security threat awareness. If users to had to rationalize a security bargain to accept fingerprint use, it will be fairly recent and better recalled, and we can also watch and compare a new security bargain being contemplated with facial recognition.

%\todo{Flynn, we need some type of bridging sentences as this section ends pretty abruptly}

