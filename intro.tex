\section{Introduction}
Biometric authentication has the potential to increase the usability of mobile
device. Frequent screen unlocking and application authorization is accomplished
with a quick glance or touch rather than recalling and entering long/complex
passcodes\cite{mayron2015biometric}. Despite the benefits, adoption can be
uneven due to usability, user misunderstanding, and security conciseness. \reminder{cite}.

From the security-conscious vantage point, in particular, allowing a new
technology to record and store a permanent signature of one's self and use it to control access to sensitive data transactions might cause deep concern. Research has documented biometric adoption \cite{bhagavatula2015biometric,rashed2015towards}, experts' sophisticated metal models of network security that are distinct from those of everyday users \cite{camp2007experimental,asgharpour2007mental}, and the influence of similar models on outlook and behavior \cite{huh2017don,das2018breaking,kang2015my,stobert2015expert,friedman2002users}. However, these studies have not specifically addressed the overlap of biometric adoption and expert understanding of the underlying
technologies and threats involved with mobile connectivity, which we have  investigated in this study. 
\reminder{cite}. 
% The potential security liabilities of tying their data security to a
% irreplaceable face or fingerprint on a vulnerable platform can be expected to
% conflict with those models of security. %CITE IF POSSIBLE
%While security familiarity has not been found to necessarily translate into textbook behavior, %CITE-
%Adoption of a new authentication mechanism offers a revealing insight into how informed users balance their expectations of safety and ease of use on mobile platforms. There is a paucity of research relating to the experiences of those with security familiarity and their perceptions of authentication types (focusing specifically on biometrics).

We seek to fill this knowledge gap by reporting on a qualitative study comparing
biometric adoption for a group of self-described security expert user and
non-expert users (n=38, 19/19 expert/non-expert). Using semi-structured
interviews, we find that there exist key differences and commonalities between
these groups that have the potential to inform the design trade-offs between
usability and security, as well as how security adoption of mobile technologies
more generally can affect a broad array of systems.

%%% This paragraph seems out of place ...
%
% %-- NON-EXPERTS MAY BE GETTING MORE SAVVY
% % First two sentences are a disclaimer that can be 86d if necessary.
% Conclusions about the security-related usability needs of security trained users
% may not transfer easily to untrained
% users. %A 2013 industry survey found that as much as 64\% of everyday users do not use any screen locking method at all, which would be difficult to reconcile with users steeped in threat awareness \cite{Consumer_survey2016}.
% The authentication behavior and outlook of these two groups diverges
% \cite{asgharpour2007mental,bravo2011bridging,schaub2012password,von2013patterns,whitten1999johnny},
% or remains unsatisfied for both groups by traditional unlocking methods
% \cite{friedman2002users,ion2015no,kang2015my}. However, there is cause to
% investigate changes in this relationship, as broader media exposure of data risk
% and security may stoke more concern, even among everyday users.  For example, a
% 2016 survey found 64\% of 18 to 26-year-old Americans recalled seeing news of a
% cyberattack in that year, almost doubling the rate of the previous year. 53\% of
% respondents in the United States also reported that cybersecurity policy
% influenced their choice of political candidates. Worldwide over the same period,
% 59\% of male young adults and 51\% of female young adults reported receiving
% formal cyber security training \cite{Raytheon_NCSA_survey2016}. In the United
% States, awareness of cybersecurity issues is also fueled by job opportunities in
% that field. The United States Bureau of Labor Statistics projected 18\% growth
% in information security jobs from 2014 to 2024, compared to 12\% for IT
% generally and 7\% for all types of employment \cite{BLS_jobStats2015}.



%--- TIMING IS GOOD
%Given the research on these subjects, we can expect to gain insight into mobile authentication adoption under differing levels of concern and understanding between everyday and security-conscious users, and differences and similarities in their behavior. However, 
We are further motivated to specifically examine adoption of biometric unlocking because fingerprint recognition, while a relatively new offering by major technology providers (iOS TouchID is the most common example in our cohort and is 5 years old at the time of this study), has gained significant consumer acceptance. %CITE EXAMPLE BAM RATE
%\todo{Flynn, I summarized the following part to save space}
As a result, many security conscious users will have recent memory of how they struck their own bargain with any doubts about its use. As facial and voice-based recognition are gaining prominence by mobile device users for authentication, our aim is to identify the security bargain made, and compare with those of fingerprints.


%Concurrently, at the time this study was conducted, facial recognition is also being broadly introduced as a common biometric authentication feature. Like fingerprint unlocking, this feature offers convenience and usability, while potentially re-triggering anxiety over security data compromise. Essentially, the timing is good to use biometrics as a subject for observing technology adoption behavior under differing levels of concern and security threat awareness. If users to had to rationalize a security bargain to accept fingerprint use, it will be fairly recent and better recalled, and we can also watch and compare a new ans similar security bargain being contemplated with facial recognition.

% Our study elicited attitudes towards the adoption of biometric authentication, specifically from an under-researched cohort of security conscious participants, and a group of non-security conscious users. A description of how individuals were categorized into these two group is presented in Section 3.

Our findings comparatively describe the outlook of two important populations of
mobile users and highlight how perspectives on adoption have evolved over
time. We also examine how factors including work requirements and device sharing
have impacted users' behavior. Our findings include:
\begin{itemize}
\item Both experts and non-experts harbored misunderstanding about biometric implementation, such as its primary vs. secondary role.
\item Both groups were subject to usability problems that led to abandoning biometric use.
\item Experts were more influenced by work/BYOD authentication requirements than non-experts.
\item Experts demonstrated more concern about securing their mobile devices
  % (e.g. describing changes to their authentication methods, expressing fear towards their biometric data leaks, and stating that security features influenced their device choices)   
\item Experts were more likely to try biometrics immediately once available, were slightly more likely to view consumer biometric security as a good idea in principle, and were more likely to recommend the features than not. 
\item Non-experts were more trusting of biometric authorization for financial applications.
\end{itemize}
Finally, these results offer insights that can
improve the design considerations for relevant authentication methods.


%%% Local Variables:
%%% mode: latex
%%% TeX-master: "main"
%%% End:
